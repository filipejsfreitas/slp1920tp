\documentclass[11pt,a4paper]{report}

%\usepackage[utf8]{inputenc}
\usepackage[portuguese]{babel}
\usepackage{graphicx}
\usepackage{url}
\usepackage{enumerate}
\usepackage{indentfirst}
\usepackage[dvipsnames]{xcolor}
\usepackage{apalike} % gerar biliografia no estilo 'named' (apalike)
\usepackage{appendix}
\usepackage{float}
\usepackage{xcolor}
\usepackage[skins,minted]{tcolorbox}
\usepackage{minted}
\usepackage[hypertexnames=false]{hyperref}
\usepackage{xspace}
\usepackage{unicode-math}
\usepackage{amsmath}
\usepackage{calc}
\usepackage{titlesec}

\titleformat{\chapter}[display]   
{\normalfont\huge\bfseries}{\chaptertitlename\ \thechapter}{20pt}{\Huge}   
\titlespacing*{\chapter}{0pt}{-50pt}{20pt}

\setminted[haskell]{xleftmargin=20pt, breaklines=true, linenos=true}

\definecolor{mintedbackground}{rgb}{0.95,0.95,0.95}
\newtcblisting{mintedhaskell}[2][]{%
    listing engine = minted,
    minted language = haskell,
    colback = mintedbackground,
    title = {#2},
    listing only,#1
}

\newtcblisting{mintedc}[2][]{%
    listing engine = minted,
    minted language = c,
    colback = mintedbackground,
    title = {#2},
    listing only,#1
}

\newtcbinputlisting{\hsfile}[3]{%
    listing engine = minted,
    minted language = haskell,
    minted options = {%
        firstline = #2,
        lastline = #3,
        autogobble = true
    },
    listing file = {src/#1},
    listing only,
    colback = mintedbackground,
    title = {\centering\makebox[\linewidth][c]{Ficheiro \textit{#1}}}
}

\graphicspath{ {./Imagens/} }

\setlength{\oddsidemargin}{-1cm}
\setlength{\textwidth}{18cm}
\setlength{\headsep}{-1cm}
\setlength{\textheight}{23cm}

\newcommand{\centertitle}[1]{\centering\makebox[\linewidth][c]{#1}}
\newcommand{\inlinebox}[1]{\framebox[\widthof{#1} + 7pt][l]{#1}}

\newcommand{\while}[0]{\textit{While}\xspace}
\newcommand{\ghci}[0]{\textit{GHCi}\xspace}
\newcommand{\data}[0]{\textit{data}\xspace}
\newcommand{\parser}[0]{\hyperref[appendix:parser]{\textit{Parser}}\xspace}
\newcommand{\classshow}[0]{\textit{Show}\xspace}
\newcommand{\classread}[0]{\textit{Read}\xspace}
\newcommand{\state}[0]{\textit{State}\xspace}
\newcommand{\smallstep}[0]{\textit{small-step}\xspace}


\title{Semântica das Linguagens de Programação \\ (3º ano da Licenciatura em Ciências da Computação)\\
       \textbf{Criação de programas em Haskell para Simulação de Semânticas}\\ Relatório de Desenvolvimento
       }
\author{Filipe Freitas (A85026)}
\date{\today}


\begin{document}

\maketitle


\tableofcontents
% \listoffigures

\setlength{\parskip}{1em}

\chapter{Introdução} \label{chapter:intro}

\section{Contextualização} \label{section:contexturalizacao}

\par Na sequência da Unidade Curricular de Semântica das Linguagens de Programação, foi proposta a realização de um trabalho que consiste na implementação, em Haskell, de vários simuladores para a linguagem \while que foi estudada nas aulas, de acordo com as semânticas naturais e operacional estruturais que foram abordadas nas aulas, bem como a implementação de simuladores para as Máquinas Abstratas AM também estudadas.

\par Sendo assim, este relatório pretende abordar as implementações criadas, explicar as decisões tomadas ao longo da implementação, e dar exemplos de programas, da sua simulação, e do resultado da sua execução.

\section{Estrutura do Relatório} \label{section:estrutura-relatorio}

\par Este relatório está dividido em 6 partes principais, e contém também 3 anexos.

\par A \hyperref[chapter:intro]{\textit{primeira}}, correspondente a esta introdução, pretende contextualizar o trabalho e explicar a sua estrutura, e explica também a estrutura do código do trabalho.

\par Na \hyperref[chapter:estruturas-linguagem-while]{\textit{segunda parte}} é feita uma introdução à linguagem \while, e são apresentadas as estruturas de dados, correspondentes à mesma, que irão servir de apoio a todo o resto do trabalho.

\par Nas \hyperref[chapter:semantica-natural]{\textit{terceira}} e \hyperref[chapter:semantica-small-step]{\textit{quarta}} partes são apresentadas implementações de semânticas naturais e operacional estruturais para a linguagem \while.

\par Na \hyperref[chapter:maquinas-abstratas]{\textit{quinta parte}} são apresentadas as máquinas abstratas AM, as suas instruções, estruturas, e correspondentes semânticas.

\par Na \hyperref[chapter:concl]{\textit{última parte}} é feita uma conclusão do trabalho realizado.

\par O \autoref{appendix:parser} contém uma explicação de um \parser para a linguagem \while, que permite a tradução de texto de entrada nas estruturas de dados internas que servem de base à linguagem \while.

\par Os Apêndices \hyperref[appendix:interpretador]{\textit{B}} e \hyperref[appendix:compilar]{\textit{C}} descrevem dois modos diferentes de execução dos vários programas: recorrendo ao interpretador \ghci ou compilando em executáveis, respetivamente.

\section{Estrutura do Código} \label{section:estrutura-codigo}

\par O código tende a estar dividido em vários ficheiros, cada um correspondente a uma área da implementação que foi pedida; também tende a separar as estruturas de dados em ficheiros individuais, para aumentar a legibilidade; e também separa cada uma das implementações de semântica e de máquina abstrata pedidas nos seus ficheiros individuais.

\par \label{paragraph:razao-tipos-explicitos} Relativamente às estruturas de dados, a maior parte são implementadas recorrendo a tipos \data ou \textit{newtype}. Esta foi uma decisão consciente de implementação, pois assim permite a declaração de instâncias \classshow e \classread dos mesmos, o que, em conjunção com o \parser que foi criado para a linguagem, torna o processo de visualização da saída do programa bastante mais simples, mesmo no \ghci. Isto tem a desvantagem de diminuir a legibilidade do código; no entanto, foram tomadas precauções para minimizar o impacto desta decisão na leitura do mesmo.

\par As implementações de instâncias \classshow são mantidas nos mesmos locais onde os respetivos tipos são declarados. No entanto, as instâncias de \classread são declaradas no ficheiro \textit{Parser.hs}, o que significa que só estão declaradas se este ficheiro tiver sido importado para a \textit{scope} atual. Mais informações sobre este detalhe são dadas no \autoref{appendix:interpretador}.

\par Para correr cada secção do programa, existe sempre um ficheiro principal que, ao ser importado, permite uma execução correta e completa de todas as funcionalidades do respetivo programa. A indicação do ficheiro principal é sempre indicada em cada secção do relatório, sempre que pertinente.

\par Em cada secção deste relatório, sempre que pertinente, é dada uma indicação de onde se pode encontrar o código ao qual o relatório está a fazer referência.

\newpage

\chapter{Estruturas da Linguagem While} \label{chapter:estruturas-linguagem-while}

\par Neste capítulo vão ser apresentadas as várias estruturas de dados que servem de suporte interno à linguagem \while.

\section{Estados} \label{section:estados}

\inlinebox{Esta secção corresponde à resolução do exercício 8. b) ii) da ficha 1.}

\par Antes de iniciar as estruturas de dados da linguagem \while propriamente ditas, vamos criar uma estrutura de dados para guardar os estados que vão ser necessários durante a interpretação de comandos da linguagem.

\par Sendo assim, e tendo em conta que, nas aulas, estados foram definidos como sendo uma função que, a cada variável, associa o seu valor inteiro correspondente, uma representação natural em Haskell desta definição é uma lista de pares, onde a primeira componente de cada par é o nome da variável, e a segunda componente é o valor da variável no estado atual.

\par Temos, assim, a seguinte definição:
\hsfile{State.hs}{7}{14}

\par Pela razão já explicada \hyperref[section:estrutura-codigo]{\textit{anteriormente}}, o tipo \state, como é visualizável na saída dos programas, é implementado com o seu próprio tipo e com o seu próprio construtor.

\par É depois definida uma instância de \classshow para o tipo \state, de acordo com a notação utilizada nas aulas:
\hsfile{State.hs}{16}{18}

\par A seguir, defini uma função para atualizar o valor de uma variável no estado, retornando um novo estado com a variável atualizada.
\hsfile{State.hs}{20}{25}

\par Defini também uma função para obter o valor de uma variável num dado estado, para simplificar algum do código que se segue.
\hsfile{State.hs}{27}{30}

\newpage


\section{Expressões Aritméticas} \label{section:expressoes-aritmeticas}

\inlinebox{Esta secção corresponde à primeira parte da resolução do exercício 8. a) i) da ficha 1.}

\par Relativamente às expressões aritméticas, nas aulas definimos que estas podem ser uma das seguintes:
$$
a ::= n \mid x \mid a_1 + a_2 \mid a_1 \times a_2 \mid a_1 - a_2
$$
\par Como uma extensão natural, decidi acrescentar também simétricos. Temos, assim, as seguintes opções para expressões aritméticas:
$$
a ::= n \mid x \mid -a \mid a_1 + a_2 \mid a_1 \times a_2 \mid a_1 - a_2
$$
\par Portanto, defini o seguinte tipo:
\hsfile{Aexp.hs}{7}{13}

\par De seguida, foi também definida uma instância de \classshow para a visualização correta das expressões aritméticas na saída da consola. Essa instância pode ser visualizada no ficheiro \textit{Aexp.hs}, linhas 18-53. A sua aparente complexidade é apenas para garantir uma saída "bonita".


\section{Expressões Booleanas} \label{section:expressoes-booleanas}

\inlinebox{Esta secção corresponde à segunda parte da resolução do exercício 8. a) i) da ficha 1.}
\\

\par No caso das expressões booleanas, nas aulas definimos que podem ser uma das seguintes:
$$
b_0 ::= \text{true} \mid \text{false} \mid a_1 = a_2 \mid a_1 \le a_2 \mid \neg b \mid b_1 \land b_2
$$
\par Por extensão natural, defini também as seguintes possibilidades para expressões booleanas:
$$
b ::= b_0 \mid a_1 > a_2 \mid a_1 \ge a_2 \mid a_1 < a_2 \mid b_1 \lor b_2
$$

\par Assim, a definição completa de expressões booleanas é:
$$
b ::= \text{true} \mid \text{false} \mid a_1 = a_2 \mid a_1 > a_2 \mid a_1 \ge a_2 \mid a_1 < a_2 \mid a_1 \le a_2 \mid \neg b \mid b_1 \land b_2 \mid b_1 \lor b_2
$$

\newpage


\par A estrutura em Haskell correspondente a esta estrutura é:
\hsfile{Bexp.hs}{9}{19}

\par Tal como acontece nas expressões aritméticas, é também depois definida uma instância de \classshow para o tipo \textit{Bexp}.


\section{Comandos da Linguagem While} \label{section:comandos-linguagem-while}

\inlinebox{Esta secção corresponde à resolução do exercício 8. b) i) da ficha 1.}
\\

\par Nas aulas definimos os comandos da linguagem \while como podendo ser um dos seguintes:
$$
S ::= \text{skip} \mid x := a \mid S_1 ; S_2 \mid \text{if}\ b\ \text{then}\ S_1\ \text{else}\ S_2 \mid \text{while}\ b\ \text{do}\ S
$$
\par Como extensão, eu acrescentei também um comando \textit{if-then}, sem \textit{else}.

\par Assim, a definição completa dos comandos da linguagem \while é:
$$
S ::= \text{skip} \mid x := a \mid S_1 ; S_2 \mid \text{if}\ b\ \text{then}\ S_1\ \text{else}\ S_2 \mid \text{if}\ b\ \text{then}\ S \mid \text{while}\ b\ \text{do}\ S
$$

\par Assim, podemos definir a seguinte estrutura para os comandos da linguagem \while:
\hsfile{Stm.hs}{9}{15}

\par Tal como nas restantes estruturas, existe também uma instância de \classshow definida para a estrutura \textit{Stm}.


\chapter{Semântica Natural} \label{chapter:semantica-natural}

\par Neste capítulo vai ser presente uma implementação de uma semântica natural para as expressões aritméticas, booleanas e comandos da linguagem \while.

\section{Expressões Aritméticas} \label{section:semantica-natural-expressoes-aritmeticas}

\inlinebox{Esta secção corresponde à primeira parte da resolução do exerício 8. a) iii) da ficha 1.}

\par A função semântica $\mathcal{A}$ é uma função que aceita expressões aritméticas, um estado, e devolve o resultado de avaliar a expressão aritmética de entrada no estado especificado. Assim, pode ser definida em Haskell do seguinte modo:
\hsfile{NS.hs}{15}{23}

\par Esta implementação é bastante simples: qualquer expressão que seja só um número retorna só esse número; se for uma variável, é só consultar o seu valor no estado; se a expressão for o simétrico de outra, então a função retorna o resultado simétrico do valor da expressão constituinte; somas, multiplicações e subtrações também são só o resultado da respetiva soma, multiplicação ou subtração das duas expressões constituintes da expressão.

\newpage

\section{Expressões Booleanas} \label{section:semantica-natural-expressoes-booleanas}

\inlinebox{Esta secção corresponde à segunda parte da resolução do exercício 8. a) iii) da ficha 1.}

\par A função semântica $\mathcal{B}$ é uma função parecida à função semântica $\mathcal{A}$. De um modo geral, podemos dizer que a função semântica $\mathcal{B}$ atua sobre expressões booleanas, em vez de expressões aritméticas. Assim, pode ser definida do seguinte modo:
\hsfile{NS.hs}{27}{40}

\section{Comandos da Linguagem While} \label{section:semantica-natural-comandos-linguagem-while}

\inlinebox{Esta secção corresponde à resolução dos exercícios 8. b) iii) e 8. b) iv) da ficha 1.}

\par É agora pedida uma função que faça a avaliação de expressões da linguagem \while. Defini essa função de acordo com as regras da semântica natural que foram apresentadas nas aulas.
\hsfile{NS.hs}{46}{62}

\par Devido ao \parser que criei, e às instâncias de \classread e \classshow criadas, testar esta função é bastante fácil. Para tal, basta, no \ghci, importar o ficheiro \textit{NS.hs}, e executar a função main. Esta espera na sua entrada um programa \while e um estado inicial, imprimindo depois o estado final, após execução do programa \while.

\par Apresenta-se a seguir a saída esperada de dois programas \while:

\begin{mintedhaskell}{\$ ghci NS.hs}
*NS> main
---------------------------------------------------------
> if x > 0 then y := x else if x < 0 then y := -x else z := 1 [x -> -1]
> if x > 0 then {y := x} else {if x < 0 then {y := -x} else {z := 1}}
  [x -> -1]
> [x -> -1][y -> 1]
---------------------------------------------------------

---------------------------------------------------------
> x := 1; while b > 0 do { x := x * a; b := b - 1 } [a -> 3][b -> 2]
> x := 1; while b > 0 do {x := x * a; b := b - 1}
  [a -> 3][b -> 2]
> [a -> 3][b -> 0][x -> 9]
---------------------------------------------------------
\end{mintedhaskell}

\par Nota: após introduzir um programa \while, a função main avalia a entrada dada, e se esta for válida, apresenta na saída o mesmo programa, sem alterações, após este ter sido convertido para as estruturas internas da linguagem \while.

\par Nota 2: Para efeitos de legibilidade, o estado é sempre impresso numa linha separada das instruções.
\\

\par Em alternativa à função main, é possível executar a função evalNS introduzindo manualmente os parâmetros do comando \while e do estado. Basta, para isso, seguir o exemplo que se apresenta a seguir:

\begin{mintedhaskell}{\$ ghci NS.hs}
*NS> let c = read "if x > 0 then y := x else if x < 0 then y := -x else z := 1" :: Stm
*NS> let s = read "[x -> 1]" :: State
*NS> evalNS (c, s)
[x -> 1][y -> 1]
\end{mintedhaskell}

\par Este processo encontra-se descrito no \autoref{appendix:interpretador}.


\chapter{Semântica Operacional Estrutural \\ (\textit{Small-Step})} \label{chapter:semantica-small-step}

\par Neste capítulo vai ser apresentada uma implementação de uma semântica \smallstep para as expressões aritméticas, booleanas e comandos da linguagem \while.

\section{Expressões Aritméticas} \label{section:semantica-small-step-expressoes-aritmeticas}

\inlinebox{Esta secção corresponde à resolução do exercício 5. a) da ficha 2.}

\par Uma semântica de transições \smallstep  para as expressões aritméticas pode ser definida do seguinte modo:
%%begin novalidate
\begin{mintedhaskell}{\centertitle{Possível implementação de stepA}}
stepA :: (Aexp, State) -> Either Val Aexp
stepA (Num n,                     _)    = Left n
stepA (Var v,                     s)    = Left $ fetch v s
stepA (Sim x,                     s)    = case stepA (x, s) of
                                              Left  x' -> Left  $ 0-x'
                                              Right x' -> Right $ Sim x'

stepA (Add  (Num a) (Num b),      s)    = Left $ a + b
stepA (Add  (Num a)      b ,      s)    = case stepA (b, s) of
                                              Left  b' -> Right $ Add (Num a) (Num b')
                                              Right b' -> Right $ Add (Num a)      b'
stepA (Add       a       b ,      s)    = case stepA (a, s) of
                                              Left  a' -> Right $ Add (Num a')     b
                                              Right a' -> Right $ Add      a'      b
                                              
(...)
\end{mintedhaskell}
%%end novalidate

\newpage

\par Este padrão da função stepA, para expressões aritméticas Add, repete-se novamente para multiplicações e subtrações. Assim, foi extraído para uma função auxiliar:
\hsfile{SOS.hs}{14}{20}

\par Assim, a implementação final da função stepA é a seguinte:
\hsfile{SOS.hs}{22}{32}

\section{Expressões Booleanas} \label{section:semantica-small-step-expressoes-booleanas}

\inlinebox{Esta secção corresponde à resolução dos exercícios 5. b) e c) da ficha 2.}

\par Uma semântica de transições \smallstep para as expressões booleanas pode ser definida de um modo análogo à semântica de transições para expressões aritméticas, com as devidas alterações. Essa função fica com o seguinte tipo: \mintinline{haskell}{stepB :: (Bexp, State) -> Either Bool Bexp}.

\par Para a avaliação de expressões base, a sua implementação é simples:
\hsfile{SOS.hs}{40}{41}

\par Para a avaliação de expressões que envolvem a comparação entre duas expressões aritméticas, podemos recorrer à função auxiliar stepAux definida acima, pois o padrão é o mesmo:
\hsfile{SOS.hs}{43}{47}

\par Para a avaliação de negações, e devido ao facto de que defini dois construtores especiais para os tipos Verdadeiro e Falso, então a implementação é um pouco mais complicada:
\hsfile{SOS.hs}{49}{53}
\par Esta implementação tem duas partes essenciais:
\begin{itemize}
    \item Parte que envolve a avaliação da negação de uma expressão booleana já totalmente simplificada (\textit{SOS.hs}, linhas 49-50) - Neste caso, a função stepB apenas retorna a negação da expressão booleana
    \item Parte que envolve a negação de uma expressão booleana não simplificada (\textit{SOS.hs}, linhas 51-53) - Neste caso, a função stepB vai executar um passo de avaliação da expressão booleana não simplificada. Se dessa execução tiver resultado um valor booleano, então a função stepB retorna uma expressão que é a negação esse valor (após utilizar a função toBexp para converter um valor booleano do Prelude para um valor booleano do tipo Bexp). Caso contrário, a função simplesmente retorna a negação da expressão não simplificada.
\end{itemize}

\par Para a avaliação de conjunções ou disjunções, a implementação tem três partes essenciais:
\begin{itemize}
    \item A avaliação de expressões onde ambos os membros estão simplificados (\textit{SOS.hs}, linhas 55-58 e 67-71) - Neste caso, apenas retornamos o resultado dessa avaliação.
    \item A avaliação de expressões onde apenas o membro direito está simplificado (\textit{SOS.hs}, linhas 59-61 e 72-74) - Neste caso, vamos executar um passo de avaliação do membro direito, retornando depois uma nova expressão que corresponde à avaliação da mesma expressão, mas desta vez com a simplificação efetuada no membro direito.
    \item A avaliação de expressões onde o membro esquerdo não está simplificado (\textit{SOS.hs}, linhas 63-65 e 75-77) - Neste caso, vamos executar um passo de avaliação do membro esquerdo. A função depois retorna uma nova expressão, que corresponde à avaliação da mesma expressão, mas com a diferença de que o membro esquerdo está um pouco mais simplificado.
\end{itemize}

\par A implementação da conjunção fica assim:
\hsfile{SOS.hs}{55}{65}

\par A implementação da disjunção é análoga. Pode ser encontrada no ficheiro \textit{SOS.hs}, linhas 66-77.
\\

\par Relativamente à implementação curto-circuito, como podemos ver, na linha 62 acima mostrada, temos um caso especial: no caso de uma conjunção onde o lado esquerdo é falso, então a função stepB retorna imediatamente o valor falso, sem avaliar o lado direito da expressão. O mesmo acontece no caso da disjunção, mas no respetivo caso (ou seja, uma disjunção onde o lado esquerdo é verdadeiro):
\hsfile{SOS.hs}{71}{71}

\newpage

\section{Comandos da Linguagem While} \label{section:semantica-small-step-comandos-linguagem-while}

\inlinebox{Esta secção corresponde à resolução do exercício 6. da ficha 2.}

\par É agora pedida a definição da função stepSOS. Esta função implementa a relação de transição $<C, s> \implies \gamma$. Como $\gamma$ pode ser tanto um estado final (se não for possível executar mais nada no programa de entrada) ou um novo programa e um novo estado ($<C, s>$), decidi, por sugestão da professora, tipar a função do seguinte modo: \mintinline{haskell}{stepSOS :: (Stm, State) -> Either State (Stm, State)}.

\par Assim, a implementação desta função apenas segue as regras da semântica \smallstep que foram ensinadas nas aulas:
\hsfile{SOS.hs}{80}{101}

\newpage

\par Com recurso à função de \textit{bind} do \textit{Monad Either}, a função nstepsSOS pode ser escrita do seguinte modo:
\hsfile{SOS.hs}{105}{106}

\par A função evalSOS também pode ser escrita em apenas uma linha devido ao \textit{binding} monádico do \textit{Either}:
\hsfile{SOS.hs}{110}{110}

\par Isto funciona pois existe a garantia de que a função stepSOS apenas retorna apenas um estado se e só se não é possível executar mais passos no programa que foi dado como entrada.

\par Novamente, para testar esta função, podemos executar com recurso ao \ghci:
\begin{mintedhaskell}{\$ ghci SOS.hs}
*SOS> main
---------------------------------------------------------
> if x > 0 then y := x else if x < 0 then y := -x else z := 1 [x -> -1]
> if x > 0 then {y := x} else {if x < 0 then {y := -x} else {z := 1}}
  [x -> -1]
> if -1 > 0 then {y := x} else {if x < 0 then {y := -x} else {z := 1}}
  [x -> -1]
> if false then {y := x} else {if x < 0 then {y := -x} else {z := 1}}
  [x -> -1]
> if x < 0 then {y := -x} else {z := 1}
  [x -> -1]
> if -1 < 0 then {y := -x} else {z := 1}
  [x -> -1]
> if true then {y := -x} else {z := 1}
  [x -> -1]
> y := -x
  [x -> -1]
> y := 1
  [x -> -1]
> [x -> -1][y -> 1]
---------------------------------------------------------
---------------------------------------------------------
> x := 1; while b > 0 do { x := x * a; b := b - 1 } [a -> 3][b -> 2]
(... Saída omitida devido ao tamanho ...)
> skip
  [a -> 3][b -> 0][x -> 9]
> [a -> 3][b -> 0][x -> 9]
---------------------------------------------------------
\end{mintedhaskell}

\par Como podemos ver, com recurso à função main, é possível consultar a saída de todos os passos individuais da execução de um programa \while.

\par No entanto, tal como descrito no capítulo anterior, também é possível executar as funções individualmente e manualmente, segundo o processo descrito no \autoref{appendix:interpretador}.

\chapter{Máquinas Abstratas} \label{chapter:maquinas-abstratas}

\par Nesta secção é descrito o processo de construção de uma implementação da semântica das três máquinas abstratas AM, AM1 e AM2.

\section{Máquina AM} \label{section:maquina-am}

\par Antes de começar a resolver o exercício 5 da ficha 3, achei pertinente implementar a máquina AM. A maior parte do código poderá depois ser facilmente adaptado para as máquinas AM1 e AM2.

\subsection{Instruções e Estrutura} \label{subsection:estrutura-maquina-am}

\par Sendo assim, comecei por criar uma estrutura correspondente a todas as instruções que a máquina AM consegue executar. Visto que estendi a linguagem \while com comandos extra, estendi também as instruções da máquina com instruções compatíveis com as extensões introduzidas na linguagem \while:
\hsfile{AMCode.hs}{7}{25}

\par Assim, código da máquina AM é representado pelo seguinte tipo:
\hsfile{AMCode.hs}{27}{27}

\par Tal como explicado anteriormente, o tipo \textit{Code} é implementado com recurso a \textit{newtype} para ser possível criar uma implementação de \classshow para o mesmo. Não existe implementação de \classread para este tipo, pois todo o código deste tipo é gerado a partir de código da linguagem \while.

\par Defini também um tipo para a \textit{Stack}:
\hsfile{Stack.hs}{5}{5}

\par O tipo \textit{Val} é definido no ficheiro \textit{State.hs}. É utilizado em todo o lado em vez de utilizar diretamente \textit{Int}, apenas para manter a consistência de tipos em todo o código. Assim, se fosse necessário alterar, essa alteração apenas tinha de ser feita num lugar.

\par A stack da máquina AM é apenas uma stack de inteiros. Isto é uma simplificação em relação ao que demos nas aulas. Valores booleanos são representados de acordo com a convenção da linguagem C:
\begin{itemize}
    \item Verdadeiro - Qualquer valor inteiro diferente de 0
    \item Falso - Valor inteiro 0
\end{itemize}

\par Assim, a máquina AM é um triplo: \mintinline{haskell}{(Code, Stack, State)}.

\subsection{Semântica} \label{subsection:semantica-maquina-am}

\par A semântica das instruções foi atribuida de acordo com o ensinado nas aulas.

\par As instruções PUSH, ADD, MULT, SUB, TRUE e FALSE são as mais simples de atribuir semântica:
\hsfile{AM.hs}{15}{20}

\newpage

\par As instruções de comparação (EQ, GT, GE, LT, LE) seguem todas a seguinte estrutura, com as devidas adaptações:
\hsfile{AM.hs}{23}{24}

\par As instruções de conjunção e disjunção seguem a seguinte estrutura, com as devidas adaptações:
\hsfile{AM.hs}{35}{36}

\par A instrução de negação também tem uma implementação simples:
\hsfile{AM.hs}{39}{40}

\par As instruções FETCH e STORE necessitam de aceder ao estado. Para isso, fazem uso das funções \textit{fetch} e \textit{update} definidas no ficheiro \textit{State.hs}:
\hsfile{State.hs}{20}{30}

\par Assim, FETCH e STORE definem-se como:
\hsfile{AM.hs}{42}{47}

\par A instrução NOOP simplesmente não atua sobre o estado nem a stack. Por ser tão trivial, não irei apresentar a sua definição.

\par A instrução BRANCH pode ser definida do seguinte modo:
\hsfile{AM.hs}{53}{55}

\par Por último, a instrução LOOP é definida à custa da instrução BRANCH, tal como ensinado nas aulas:
\hsfile{AM.hs}{58}{58}

\newpage

\subsection{Compilador de programas \while} \label{subsection:compilador-maquina-am}

\par Um compilador é, neste contexto, apenas uma função com o seguinte tipo: \mintinline{haskell}{cmpS :: Stm -> Code}.

\par Mas antes de ser possível compilar programas \while completos, temos de ser capazes de compilar expressões aritméticas. Assim, a compilação de expressões aritméticas dá-se do seguinte modo:
\hsfile{AM.hs}{71}{79}

\par A função cmpAux apresentada é uma abstração de um padrão comum que se repete várias vezes ao longo do compilador, onde as instruções geradas têm o seguinte padrão: op2:op1:OP, onde op2 são as instruções que levam à produção do operando 2 no topo da stack, op1 são as instruções que levam à produção do operando 1 no topo da stack, e OP é a instrução corresponde à operação que se pretende efetuar (como por exemplo uma soma). Está definida assim:
\hsfile{AM.hs}{66}{68}

\newpage

\par Também temos de ser capazes de compilar expressões booleanas:
\hsfile{AM.hs}{82}{97}

\par Podemos agora compilar qualquer expressão da linguagem \while:
\hsfile{AM.hs}{100}{115}

\newpage

\subsection{Função Semântica $S_{am}$} \label{subsection:funcao_semantica_am}

\par A função semântica $S_{am}$ pode ser definida do seguinte modo:
\hsfile{AM.hs}{118}{120}

\par Esta função pode, como de costume, ser testada no \ghci:
\begin{mintedhaskell}{\$ ghci AMRepl.hs}
*AMRepl> sAM (read "while x > 0 do x := x - 1") (read "[x -> 3]")
[x -> 0]
\end{mintedhaskell}

\newpage

\section{Máquina AM1} \label{section:maquina-am1}

\inlinebox{Esta secção corresponde à resolução dos exercícios 5. a), b) e c) da ficha 3.}

\subsection{Memória} \label{subsection:estrutura-memoria}

\par A arquitetura da máquina AM1 substitui o estado por uma memória, análoga à memória de um computador tradicional: é apenas uma grande lista de valores inteiros. Assim, defini uma estrutura especializada para esta memória, bem como funções para a manipular:
\hsfile{Memory.hs}{5}{6}
\hsfile{Memory.hs}{13}{24}

\par A função get é bastante simples: apenas consulta na memória o valor do endereço que foi passado de entrada.

\par A função put já é mais complexa: dado um endereço e um valor, tem de colocar na memória esse valor, mas no endereço certo. Se, por acaso, a lista que representa a memória for demasiado pequena, então a função put tem de aumentar o tamanho da lista. Cada caso especial é tratado individualmente na função.

\subsection{Instruções e Estrutura} \label{subsection:estrutura-maquina-am1}

\par Segundo o pedido no exercício, a máquina AM1 deixa de ter as instruções FETCH-x e STORE-x, passando a ter, respetivamente, as instruções GET-n e PUT-n.

\par Isto é representado na lista de instruções da máquina AM1 pelas seguintes alterações:
\hsfile{AM1Code.hs}{22}{23}

\subsection{Semântica} \label{subsection:semantica-maquina-am1}

\par Mantém-se a maior parte da semântica anteriormente atribuída. No entanto, foram removidas as implementações de semântica relativas às instruções FETCH e STORE, sendo introduzida semântica para as instruções GET e PUT:
\hsfile{AM1.hs}{44}{49}

\subsection{Compilador} \label{subsection:compilador-maquina-am1}

\par O compilador desta máquina é, geralmente, igual ao compilador da máquina AM. No entanto, o compilador de AM1 não aceita apenas um programa \while; aceita também um Estado. Um estado é uma abstração conveniente que já tinha sido anteriormente criada, e que se adequa perfeitamente às nossas necessidades: este estado será responsável por associar às variáveis do programa \while localizações na memória.

\par A função de atribuição de endereço a uma variável é simples:
\hsfile{Memory.hs}{40}{44}

\par Esta função é simples: consulta o estado à procura da variável pedida. Se por acaso essa variável não existir, então atribui-lhe um endereço: esse endereço é simplesmente o número de variáveis já com endereço atribuído + 1.

\par Assim, para compilar Assignments da linguagem \while, é necessário ter em conta os endereços:
\hsfile{AM1.hs}{107}{110}

\par Nas restantes equações das funções de compilação, é apenas necessário ter cuidado para gerir corretamente o estado, atualizando-o sempre que necessário, ou sempre que possam ter ocorrido alterações no mesmo em execuções recursivas da função de compilação.

\subsection{Função Semântica $S_{am1}$} \label{subsection:funcao_semantica_am1}

\par A função semântica $S_{am1}$ pode ser definida do seguinte modo:
\hsfile{AM1.hs}{130}{132}

\par Esta função pode, como de costume, ser testada no \ghci:
\begin{mintedhaskell}{\$ ghci AM1Repl.hs}
*AM1Repl> sAM1 (read "while x > 0 do x := x - 1") (setupMemory (read "[x -> 3]") (read "[x -> 0]") (Memory []))
0
\end{mintedhaskell}

A função setupMemory recebe dois estados: o primeiro é o estado inicial das variáveis do programa While; o segundo é a associação entre o nome de uma variável e o seu endereço de memória. O terceiro argumento é para permitir uma recursividade na definição da função; basta ser inicializado como vazio.

\newpage

\section{Máquina AM2} \label{section:maquina-m2}

\inlinebox{Esta secção corresponde à resolução dos exercícios 5. d), e) e f) da ficha 3.}

\subsection{Instruções e Estrutura} \label{subsection:estrutura-maquina-am2}

\par A máquina AM2 introduz, em relação à AM1, um \textit{program counter} para que não seja necessária modificação do código da máquina. Esta máquina substitui também as instruções BRANCH() e LOOP() por instruções de LABEL-$\ell$, JUMP-$\ell$ e JUMPFALSE-$\ell$. Assim, a semântica desta máquina nunca modifica o código da mesma; apenas modifica o \textit{program counter}.

\par As instruções JUMP-$\ell$ e JUMPFALSE-$\ell$ saltam para o endereço onde se encontra a instrução LABEL-$\ell$. Sendo assim, quando encontra uma destas instruções, a máquina tem pelo menos duas opções:
\begin{enumerate}
    \item Procurar em todo o código onde se encontra o LABEL pretendido;
    \item A própria instrução JUMP-$\ell$ e JUMPFALSE-$\ell$ pode incluir o endereço relativo da instrução LABEL-$\ell$ pretendida.
\end{enumerate}
\par A minha implementação opta pela segunda opção.

\par Sendo assim, as novas instruções introduzidas são:
\hsfile{AM2Code.hs}{25}{27}

\newpage

\subsection{Semântica} \label{subsection:semantica-maquina-am2}

\par A semântica da maior parte das instruções foi alterada para não modificar o código, mas sim o \textit{program counter}.

\par Às instruções de JUMP e JUMPFALSE foi-lhes atribuida a seguinte semântica:
\hsfile{AM2.hs}{63}{68}

\subsection{Compilador} \label{subsection:compilador-maquina-am2}

\par O compilador da linguagem \while para os comandos If-Then-Else, If-Then, e While tiveram de ser alterados para acomodar as novas instruções.

\par O número do último \textit{label} atribuído é guardado no Estado que guarda os endereços das variáveis, sob o nome "\$LABEL". Sempre que é necessário criar um novo \textit{label}, o estado é consultado para saber qual deve ser o próximo número de \textit{label} a atribuir.

\par No caso do If-Then-Else, foi criado um \textit{label} no código que é o corpo do Then e o corpo do Else. As instruções geradas são, portanto, a avaliação da condição b; depois um JUMPFALSE, que, se b for falso, salta para o corpo do Else, recorrendo ao \textit{label} lá colocado. caso contrário, a execução continua normalmente.
\hsfile{AM2.hs}{128}{135}

\newpage

\par O caso do While contém dois \textit{labels}: um dos \textit{labels} corresponde ao código que se encontra APÓS TODO o ciclo While; outro \textit{label} encontra-se exatamente no início do ciclo While. O While é depois implementado com dois saltos: um JUMPFALSE que salta para o \textit{label} final, no caso de a condição do While ser falsa; e, no final da execução do corpo do While, encontra-se um JUMP incondicional que salta sempre para o \textit{label} que se encontra no início de todo o ciclo While.
\hsfile{AM2.hs}{139}{145}

\subsection{Função semântica $S_{am2}$} \label{subsection:funcao_semantica_am2}

\par A função semântica $S_{am2}$ é idêntica à função semântica induzida por AM1:
\hsfile{AM2.hs}{151}{153}

\par Esta função pode, como de costume, ser testada no \ghci:
\begin{mintedhaskell}{\$ ghci AM2Repl.hs}
*AM2Repl> sAM2 (read "while x > 0 do x := x - 1") (setupMemory (read "[x -> 3]") (read "[x -> 0]") (Memory []))
0
\end{mintedhaskell}

\chapter{Conclusão} \label{chapter:concl}

\par Este trabalho permitiu uma profunda consolidação dos conhecimentos sobre todas as semânticas que estudamos ao longo das aulas da UC; permitiu também uma profunda consolidação de conhecimentos sobre a linguagem de programação Haskell.

\appendix \label{appendix}


\chapter{Parser para a linguagem While} \label{appendix:parser}

\par No ficheiro \textit{Parser.y} está incluido um \parser para a linguagem \while que estudamos, bem como a notação que utilizamos para os estados das variáveis. Este \parser é baseado numa gramática independente de contexto, e utiliza a \textit{library} Happy para o seu funcionamento.

\par A sua compilação num ficheiro Haskell é simples: após instalação, basta executar o comando \textit{happy Parser.y}. Por conveniência, é já incluído um ficheiro Parser.hs, derivado a partir da descrição da gramática, no ficheiro ZIP que contém todo o trabalho.

\par Apenas com a inclusão do \parser é que são declaradas instâncias de \classread para os tipos da linguagem \while; se este \parser não for incluido, não é possível recorrer às instâncias de \classread.

\chapter{Como correr os vários programas no interpretador GHCi} \label{appendix:interpretador}

\par Para correr um dos programas no \ghci é só necessário carregar o ficheiro correto no \ghci. Cada semântica tem o seu ficheiro principal:

\begin{itemize}
    \item Semântica Natural - NS.hs
    \item Semântica Small-Step - SOS.hs
    \item Máquina AM - AMRepl.hs
    \item Máquina AM1 - AM1Repl.hs
    \item Máquina AM2 - AM2Repl.hs
\end{itemize}

\par Se for incluido um destes ficheiros, é sempre possível chamar a função main, que inicia um Read-Eval-Print-Loop (REPL), que espera, no stdin, por programas da linguagem \while, e depois, dependendo do programa carregado, executa os vários passos de execução desse programa de acordo com a semântica que está carregada, ou no caso das máquinas, compila o código, e executa na respetiva máquina.

\par A execução de qualquer um destes ficheiros também incluí todos os símbolos necessários à construção de expressões para a respetiva semântica e/ou máquina.

\chapter{Como compilar os vários programas com recurso à Makefile} \label{appendix:compilar}

\par Com recurso à Makefile, é possível executar \textit{make} para compilar um executável por programa.

\par Podem ser compilados os seguintes programas:

\begin{itemize}
    \item make NS
    \item make SOS
    \item make AM
    \item make AM1
    \item make AM2
\end{itemize}

\par A execução de um programa compilado deste modo funciona do mesmo modo do que a execução da função main descrita no \autoref{appendix:interpretador}.

\end{document}
